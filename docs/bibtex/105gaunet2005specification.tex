@article{105gaunet2005specification,
author = {Florence Gaunet and Xavier Briffault},
title = {Exploring the Functional Specifications of a Localized Wayfinding Verbal Aid for Blind Pedestrians: Simple and Structured Urban Areas},
journal = {Human–Computer Interaction},
volume = {20},
number = {3},
pages = {267--314},
year = {2005},
publisher = {Taylor \& Francis},
doi = {10.1207/s15327051hci2003\_2},
URL = { https://www.tandfonline.com/doi/abs/10.1207/s15327051hci2003_2 },
eprint = { https://www.tandfonline.com/doi/pdf/10.1207/ },
We found that specific database features are streets, sidewalks, crosswalks, and intersections and that guidance functions consist of a combination of orientation and localization, goal location, intersection, crosswalks, and warning information as well as of progression, crossing, orientation, and route-ending instructions; they have to be provided between 5 to 10 meters before an intersection, after crossing, at middle block, and after entering a street. Last, verbal guidance is possible in simple and structured urban areas, with no localization aid, and is optimal within 5 meters' precision. The outcomes and limits of the requirements of the navigational aid evidenced are discussed. }
}